\chapter{Tracks}
\begin{multicols*}{2}
Tracks come in a variety of different forms, however, a lot of common features can be noted. All tracks are either Major, Minor or Passive, and either Offensive, Defensive, Mixed or Varies. In this section, we give an overview of each type of track.

Major tracks determine how you use your major action. These tracks offer powerful options to use in place of basic attacks. Each character requires one of these. Examples include Compensator, which allows you to wield powerful weapons and Human Shield, which allows you to take powerful defensive stances to protect your allies and heal them. These tracks also give you your triggered strikes and may provide boosts to your charge attacks. As your major action is worth more than your other actions, these tracks often are about twice as powerful as other tracks, in exchange for requiring the use of your major action.

Minor tracks determine how you use your minor action. These tracks offer options to use in place of default minor actions. Each character requires one of these. Examples include Breath Weapon, allowing you to breath fire or other elements, and Shield, which allows the player to block incoming damage.

Passive tracks generally do not use actions, but may influence your movement. Once again, each character requires one of these. Examples include Death Armor, which allows you to protect yourself with the souls of your fallen enemies, and Ninja, which allows you to hide from your enemies.

Offensive tracks are generally focused on damaging your enemies, however, they can also amplify the damage-dealing abilities of your allies. Each party should have several players with at least one offensive track, however, a character without any offensive tracks is viable as a support character. Examples of offensive tracks include Elementalist, which allows you to throw powerful magic at your enemies, Two Weapon Fighting, which allows you to make off-hand attacks, and Fire Elemental, which burns creatures around you.

Defensive tracks are generally focused on keeping you and your allies alive. This can be done by giving you more HP or recovering HP you lose, or by reducing the ability of enemies to damage you. In some occasions, it can negate an enemy attack entirely. It is highly recommended that a player has at least one defensive track. A character with 3 defensive tracks is viable, however, they will exist mostly as support. Major defensive tracks will often also deal damage as well as provide defenses, however, they should not be the only form of damage that the party can deal.

Mixed tracks are tracks which contain some combination of offense and defense. While other tracks provide pure offense or pure defense, mixed tracks allow the player to adjust to the specific situation, generally at the cost of the more tactical abilities. Examples include the polymorph track, which allows the player to take on forms with potent offensive or defensive abilities as the situation requires, the potions track, which gives access to a wide variety of potions such as potions of might, potions of healing and potions of haste. A mixed track is not required. More experienced players can combine multiple mixed tracks to form a valid combination of offense and defensive, however, this is a difficult character to play well, and may often find himself dealing insufficient damage, or having insufficient ways to protect himself.

Varied tracks are tracks that have permanent choices that change the role or function of a track - unlike Mixed, they can’t adapt to the situation, but typically can cover their chosen role at full potency. A character can use a varied track as an offensive or defensive track by taking abilities that are mostly offensive or mostly defensive.

\end{multicols*}

%Major
\section{Brawler}
\textbf{Major Offensive}

Cause bodily harm to your enemies with your body, up close and personal.

\subsection*{Adaption}
This could be a form of martial arts, brawling, cybernetic enhancements, or raw animal fury.


\begin{multicols*}{2}
\subsection*{First \Facet\perk}
\textbf{One-Two} [WO]: Your unarmed attacks count as being armed with a melee weapon. As a major action, make two pool attacks dealing $10|13|17|20 \times Lvl$. \textbf{Boost}: Gain a +3 bonus to hit.

\textbf{Flying Tackle} [HO]: As a major action, dash up to your speed in a straight line and attack an adjacent opponent at the end of your movement with a +2 to hit. This deals $3\times C\times Lvl$ damage and you begin grappling them if you hit. If you moved at least 3 hexes with this action, that enemy becomes prone if you hit, and you inflict $2\times C\times Lvl$ to the target regardless of whether or not you hit. \textbf{Boost}: You deal an additional $\frac{3}{2} \times C\times Lvl$ damage if you hit.

\textbf{Jab} [W]: When you use a surge, make a single attack dealing $10|13|17|20 \times Lvl$ damage. This counts as a single attack with One-Two.

\subsection*{Second \Facet\perk}
\textbf{Give 'em the Chair!} [WO]: As a major action while wielding an improved weapon, make a single attack with a +1 to hit that deals $6 \times C \times Lvl$ and inflicts [Senseless; Stunned]. If you hit, the improvised weapon receives the same amount of damage, potentially destroying it.  \textbf{Boost}: Gain an additional +3 bonus to hit.

\textbf{Suplex} [O]: As a major action while grappling an opponent, you may cease grappling them. If you do, that enemy is moved into an adjacent unoccupied hex, becomes [Prone] and receives $4 \times C\times Lvl$ damage. \textbf{Boost}: Deal an additional $C\times Lvl$ damage.

\subsection*{Third \Facet\perk}
\textbf{Disorienting Blows}: One-Two and Give 'em the Chair inflict [Senseless; Off Balance].

\subsection*{Fourth \Facet\perk}
\textbf{Steel Grip}: The defense roll to escape your grapples has +2 potency.

\subsection*{Fifth \Facet\perk}
\textbf{Terrain Rip}: If you do not target a creature with One-Two and destroy a piece of [Destructible] terrain, you may instead begin wielding it as an improved weapon and immediately make an attack with Give 'em the Chair!

\subsection*{Sixth \Facet\perk}
\textbf{Throw}: Suplex may move the opponent to any unoccupied hex within range 3.

\subsection*{Seventh \Facet\perk}
\textbf{Concussion}: One-Two and Give 'em the Chair inflict [Blind 8] for the rest of the encounter. If you would attack a creature already affected by this ability, instead reduce the radius of this blinded effect by 1.

\end{multicols*}

\section{Ying}
\textbf{Major Defensive}

You channel both the rift of death and the rift of life. While one may seem to be dominant, both always work in harmony in your abilities.

\subsection*{Adaption}
This can represent any kind of healer, whether they are casting spells, imparting the favor of their gods, or shooting vials of adrenaline into people.

\begin{multicols*}{2}
\subsection*{First \Facet\perk}
\textbf{Cure Wounds} [RmS]: As a major action, you may heal an ally within twelve hexes for $2 \times C \times Lvl$. When you do so, you may add one bonus to them. \textbf{Boost}: Heal an additional $C \times Lvl$.

\textbf{Energiser}: You start with the energiser bonus. When you use this bonus, the target of your cure wounds is [Energized $2 \times C \times Lvl$].

\textbf{Inflict Wounds} [RmS]: As a major action, choose an enemy within 12 hexes. That enemy makes a defence roll. If they fail the defence roll, deal 1d4 damage for every level you possess, multiplied by your caliber to that enemy. If they instead succeed, deal half that damage instead. Regardless of success or failure, add a hindrance to the targeted enemy. This ability has any tags that the hindrance has. \textbf{Boost}: After the effects of the hindrance, deal an additional $C \times Lvl$ damage. This damage is not halved on a successful defence roll.

\textbf{Weakening Influence} [O]: You start with the weakening influence hindrance. When you use this hindrance, the target of your inflict wounds is [Weakened X], where X is the amount of damage dealt by the inflict wounds ability.

\textbf{Surge} [Rm]: As a surge, you may use the inflict wounds ability. You may not add a hindrance when using it this way.


\subsection*{Second \Facet\perk}
\textbf{Relief}: When you use cure wounds on an ally, you may also choose a [Restorable; Lesser] condition affecting that ally. If you do, end that condition.

\textbf{Extended Protection}: If you heal a damaged ally to more than their maximum hit points, the excess becomes temporary HP.

\textbf{Flaming Sword}: You gain the flaming sword bonus. When you use this bonus, choose an [Energy Condition]. The target of your cure wounds may attach a rider to the first major action they take during their next three turns. This rider deals $C \times Lvl$ damage on a hit and the target receives the chosen [Energy Condition]. If the rider misses, the target of the attack still takes half damage.

\textbf{Vampiric} [O]: You gain the Vampiric Wounds hindrance. When you use this hindrance, you heal yourself an amount equal to the damage dealt by inflict wounds.

\subsection*{Third \Facet\perk}
\textbf{Protective Aura}: You gain the Protective Aura bonus. When you use this bonus, the next time that the affected creature would be damaged, the creature who damaged the affected creature takes damage equal to the amount healed by cure wounds.

\textbf{Doom}: You gain the Doom hindrance. When you use this hindrance, the target of your inflict wounds ability gains [Persistent $2\times C\times Lvl$] if they fail the defense roll.

\subsection*{Fourth \Facet\perk}
\textbf{Vengeance}: You gain the Vengeance bonus. When you use this bonus, the targeted ally may move up to three hexes and use a surge against an opponent that attacked them in the last round.

\textbf{Parasite}: You gain the Parasite hindrance. When you use this hindrance, the target of your inflict wounds ability gain an equipped effect with $2 \times C \times Lvl$ non-bonded HP.  While the equipped effect remains alive, the affected creature is [Slowed 3].

\subsection*{Fifth \Facet\perk}
\textbf{Desperate Cure}: You may use cure wounds as an immediate major action.

\textbf{Quick Death}: You may use inflict wounds as an immediate major action.

\textbf{Healing Surge}: You gain the Healing Surge bonus. When you use this bonus, the target of your cure wounds ability may take a minor action.

\textbf{Hammer}: You gain the Hammer hindrance. When you use this hindrance on an enemy, if that enemy is making an attack, make an attack with a -2 penalty to hit. If hit hits, the enemy is moved 1 hex per odd circle before their attack (potentially wasting their action). If the attack misses, the movement still occurs, but it occurs after their attack.

\subsection*{Sixth \Facet\perk}
\textbf{Glory of the Fallen}: \emph{Your healing is so powerful it can revive wounded allies.} You may use cure wounds on a [Wounded] creature. If this brings them above 0 hp, they are no longer wounded.

\textbf{Damage Aura}: You gain the Damage Aura bonus. When you use this bonus, the target of your cure wounds ability deals $\frac{1}{2} \times C \times lvl$ damage to each creature within their melee range at the beginning of each of their turns for the rest of the encounter.

\textbf{Weakening Burst}: You gain the Weakening Burst hindrance. When you use this hindrance, the target of your inflict wounds and each enemy adjacent to them is [Weakened $-|-|7|8 \times C \times Lvl$].

\subsection*{Seventh \Facet\perk}
\textbf{Lifelink}: You may use cure wounds on your allies even if you do not have line of effect to them. They must still be within range 12 of you.

\textbf{Etheral}: You gain the Ethereal bonus. When you use this bonus, the target of your cure wounds has [Phasing] and a lowered target priority for one round, and they gain two boosts. Attacks against this target for one round that violate target priority receive a penalty.

\textbf{Binding}: You gain the Binding hindrance. When you use this hindrance, the target of your inflict wounds is [Slowed 3] for one round and receive two penalties.
\end{multicols*}

\section{Alchemist}
\textbf{Major Mixed}

You combine reactants which, individually, may be weak, but together, combine for a flexible array of powerful effects.

\subsection*{Adaption}
This track is, by default, flavoured as mixing ingredients, however, it can instead be fluffed as making your own spells.

\begin{multicols*}{2}
\subsection*{First \Facet\perk}
\textbf{Recipes}: At the beginning of each encounter, make a number of recipes equal to your Caliber. Each recipe consists of either two main components, or a main component and a secondary component. You begin the encounter with those recipes prepared. The list of components can be found on page \hyperref[alchposmain]{\pageref{alchposmain}}.

\textbf{Mystery Mixture} [RmOS]: As a major action, use a recipe that you have prepared on a creature within 6 hexes of you. That recipe stops being prepared when you use it. \textbf{Boost}: If you use this ability on an ally, that ally becomes [Energized CLvl]. If you use this ability on an enemy, that enemy becomes [Vulnerable $C\times Lvl$].

\textbf{Fetch Components}: You may re-prepare each recipe that you have used this encounter by spending 3 movement points. Each recipe becomes prepared again.

\textbf{Surge}: As a surge, you may use one of the main components of a recipe that you prepared at the beginning of the encounter.

\subsection*{Second \Facet}
\textbf{Ingredients}: When you gain this ability, choose two ingredients. These ingredients form your ingredient list. When you gain another \facet of this track, you may add another two ingredients to this list (resulting in a total of 12 ingredients in your ingredient list when you gain your 7th \facet). When you make a recipe, you may add an ingredient from your ingredients list to that recipe. The available ingredients can be found on pages \hyperref[alchposing]{\pageref{alchposing}}.

\subsection*{Third \Facet\perk}
\textbf{Alchemical Ammunition} [RmOS]: As a major action, you may choose one of your recipes and an adjacent ally. The next time that ally deals damage with a weapon attack, apply the effects of the recipe to the damaged creature. (Note: While the name of this ability refers to ammunition, it works equally well with damage from a melee weapon as it does with ranged damage.)

\subsection*{Fourth \Facet}
\textbf{Increased Ingredients}: When making a recipe, you may use two ingredients from your ingredients list instead of one. You cannot choose the same ingredient twice.

\subsection*{Fifth \Facet\perk}
\textbf{Reckless Recipe}: When making a recipe, you may use an additional main or secondary component. When you use that recipe, you become [Vulnerable $2\times C \times Lvl$].

\subsection*{Sixth \Facet}
\textbf{Incredible Infusion}: When making a recipe, you may use three ingredients from your ingredients list instead of two. You cannot choose the same ingredient more than once

\subsection*{Seventh \Facet\perk}
\textbf{Spontaneous Preparation}: Once per encounter, you can make a recipe and add it to your recipe book. You may only use that recipe once. \emph{(Tip: Prepare it during other people’s turns)}

\subsection*{Positive Main Components}
\label{alchposmain}
These components are best used with other positive components and ingredients, and used on allies.
\begin{itemize}
\thing \textbf{Liquid Luck}: At the beginning of the targeted creature’s next three turns, they receive a boost.
\thing \textbf{Strengthening Stimulants}: The target of this effect becomes [Energized $2\times C\times Lvl$].
\thing \textbf{Potion of Perception}: The next time that the target of this effect would make an overwhelming attack that has a 50\% miss chance or less, ignore that miss chance. Alternatively, the next two times that the target of this effect would make a non-overwhelming attack that has a 50\% miss chance or less, ignore that miss chance.
\thing \textbf{Rapidness Root}: During the target of this effects next turn, they may use a surge as an additional action.
\end{itemize}

\subsection*{Positive Secondary Components}
\label{alchpossec}
These components are best used with other positive components and ingredients, and used on allies.
\begin{itemize}
\thing \textbf{Healing Herbs}: The target of this effect gains $2\times C\times Lvl$ hit points. Any hit points healed above their maximum are converted to temporary HP.
\thing \textbf{Refreshing Restorative}: The target of this effect gains a refresh.
\thing \textbf{Common Cure-all}: The target of this effect selects one [Restorable; Greater] effect currently affecting them. That effect ends.
\thing \textbf{Drink of Defense}: The target of this effect receives [Persistent Guarded 1] for three rounds.
\end{itemize}

\subsection*{Positive Ingredients}
\label{alchposing}
These components are best used with other positive components and ingredients, and used on allies.
\begin{itemize}
\thing \textbf{Invisible Ingredient}: The creature affected by this ability becomes [Invisible 4] until the end of their next turn.
\thing \textbf{Dash of Dashing}: At the beginning of the affected creature’s next turn, that creature may dash 2 hexes.
\thing \textbf{Enlarging Elixir}: Until the end of your next turn, the affected creature threatens hexes one hex further away than usual.
\thing \textbf{Restorative Reagents}: The target of this effect selects one [Restorable; Lesser] effect currently affecting them. That effect ends.
\thing \textbf{Aiming Additive}: The affected creature may treat creatures and hexes up to 12 hexes away as if they were within 6 hexes for the purpose of targeting with abilities.
\thing \textbf{Bit of a Bite}: The affected creature gains a natural weapon for one round.
\thing \textbf{Slice of Style}: During the affected creature’s next turn, it may use a Combat Maneuver as an additional action.
\thing \textbf{Pinch of Protection}: Until the beginning of your next turn, the affected creature’s Target Priority is increased by two stages. When an enemy would violate target priority by not attacking the affected creature, the affected creature can use a surge against the attacking enemy.
\thing \textbf{Fading Fragrance}: Until the beginning of your next turn, the affected creature’s Target Priority is decreased by two stages. When an enemy would violate target priority by attacking the affected creature, that attack receives a penalty.
\end{itemize}

\subsection*{Negative Main Components}
\label{alchnegmain}
These components are best used with other negative components and ingredients, and used on enemies.
\begin{itemize}
\thing \textbf{Poisonous Powder}: The target of this effect receives [Ongoing $C\times Lvl$] for 3 rounds.
\thing \textbf{Distilled Disadvantage}: The target of this effect becomes [Persistent Off Guard 1].
\thing \textbf{Vial of Vulnerability}: The target of this effect becomes [Vulnerable $2\times C\times Lvl$].
\thing \textbf{Supplement of Solitude}: The target of this effect becomes [Battered $3\times C\times Lvl$].
\end{itemize}

\subsection*{Negative Secondary Components}
\label{alchnegsec}
These components are best used with other negative components and ingredients, and used on enemies.
\begin{itemize}
\thing \textbf{Flavoring of Frailty}: The target of this effect becomes [Weakened $2\times C\times Lvl$].
\thing \textbf{Ounce of Obliviousness}: The target of this effect receives two penalties.
\thing \textbf{Steeped in Slowness}: The next time the target would make a surge, they instead don’t.
\thing \textbf{Double Vision}: The next two times that the target attacks any creature, that creature gains a 50\% miss chance against that attack. If this attack is overwhelming, reduce the miss chance to 25\%.
\end{itemize}

\subsection*{Negative Ingredients}
\label{alchneging}
These components are best used with other negative components and ingredients, and used on enemies.
\begin{itemize}
\thing \textbf{Drop of Distraction}: The affected creature becomes [Off Balance].
\thing \textbf{Elemental Enzyme}: When you make a recipe with this ingredient, choose an elemental status. When you use this ingredient, the affected enemy receives the chosen elemental status.
\thing \textbf{Sprinkling of Sightlessness}: The creature affected by this ability becomes [Blinded 5] for one round.
\thing \textbf{Tad of Torment}: The creature affected by this ability becomes [Vulnerable $C\times lvl$] to the next reactive strike made against them.
\thing \textbf{Grounded Gravity}: At the end of the affected creature’s next turn, that creature loses the ability to fly for one round and cannot gain the ability to fly for one round. 
\thing \textbf{Vapours of Vision Violation}: The creature affected by this ability loses [Scanner] until the end of their next turn, and cannot gain [Scanner] through any means.
\thing \textbf{A Light Touch of Light}: All creatures have line of sight to the affected creature until the beginning of your next turn. All creatures gain line of sight to you for one round when you use this ingredient.
\thing \textbf{Confusing Condiments}: The next time that the affected creature would take a major action, randomly choose an ally within range of that attack. The affected creature treats that ally’s target priority as elevated for that action, and treats all other ally’s target priority as lowered for that action. That enemy cannot violate target priority with that attack.
\thing \textbf{Splash of Subtlety}: Allies may treat the affected creature’s target priority as elevated or lowered, at their discretion.
\end{itemize}

\end{multicols*}

\section{Summoner}
\textbf{Major Mixed}

Some people don't fight themselves, and don't rely on their allies. Instead, they summon forth creatures to fight on their behalf. Summoners comes from all walks of life, from the shaman who summons forth spirits of nature, to the occultist who binds demons to his will. Regardless of the source, summoners call upon a variety of creatures to do their bidding.
\subsection*{Adaption}
This track is written to represent summoned creatures, but can also represent short-lived automatons or mobile spell effects.

\begin{multicols}{2}
\raggedcolumns
\subsection*{First \Facet}
\textbf{Summon}: You may create a list of possible creatures to summon. It has 1 creature plus 1 creature per circle on it. Each summon has $2 \times C \times Lvl$ HP and defense and offense values matching your own. Each one knows two mook major abilities. Additionally, it may either have a natural weapon or a wielded weapon and a spare weapon.

\textbf{Link}: When one of your summons would be healed, you may choose to have that healing affect you instead.

\textbf{Summoning Style}: Choose either Fixed Duration Summon or Variable Duration Summon. This choice is permanent:
\begin{itemize}
\thing \textbf{Fixed Duration Summon} [R]: As a major action, summon forth a creature from your list. When you summon it, and at the beginning of your next turn, if the creature is still alive, it may move up to 6 hexes and use one of the mook major abilities that it knows. After those two attacks, the summon is dismissed. It can use its mook major abilities as a triggered strike. \textbf{Boost}: When you boost this ability, the summoned creature gains a boost on its next attack.
\thing \textbf{Variable Duration Summon} [R]: As a major action, summon forth a creature from your list. When you summon it, it may move up to 6 hexes and use one of the mook major abilities that it knows. At the beginning of each of your turns, there is a 50\% chance that your summoned creature disappears. If it does not, the summon can move up to 6 hexes and use one of its mook major abilities. \textbf{Boost}: When you boost this ability, the summoned creature gains a boost on its next attack.
\end{itemize}
\textbf{Point-blank Summoning}: When you summon a creature, you may choose an enemy who you are in the melee range of. if the summoned creature is summoned in a hex adjacent to that enemy, the act of summoning that creature does not provoke reactions, but the summoned creature must attack the chosen enemy.

\textbf{Surge}: When you would use a surge, make an attack against an enemy. If you hit, that enemy becomes [Vulnerable $5|7|8|10\times lvl$] to the next attack made by a summoned creature you control.

\subsection*{Second \Facet}
\textbf{Customised Summons}: Each summon on your list gains a tactical ability or two features.

\textbf{Extended Summons}: If you possess the Fixed Duration Summon ability, when you would use a surge that is not a reactive strike, instead, choose a summon you have active. That summon gains $2\times C\times Lvl$ THP, and the next time that that summon would be dismissed, it is instead dismissed one round later.

If you possess the Variable Duration Summon ability, when you would make a triggered strike that is not a reactive strike, instead, choose a summon you have active. That summon gains $2\times C\times Lvl$ THP, and at the beginning of your next turn, do not check if that summon disappears that turn.

\subsection*{Third \Facet\perk}
\textbf{Natural Power}: Each summon on your list gains one of the following abilities:
\begin{itemize}
\thing \textbf{Destructive Aura}: At the beginning of your turn, every enemy adjacent to this summon takes $C\times Lvl$ damage. A creature may dash 1 hex away from the summon to prevent this damage.
\thing \textbf{Skitter}: Once per round, after being attacked, this summon may dash 3 hexes.
\thing \textbf{Restorative}: Once per round, you may remove a [Restorable; lesser] effect from this summon. 
\thing \textbf{Web}: Once per round, this summon may create a 1 hex burst within 6 hexes that inflicts [Slow 1] to any creature entering it.
\thing \textbf{Venomous}: You may attach a rider to each of this summon’s mook major actions that inflict [Persistent; dizzy 1] on a hit.
\thing \textbf{Energy Attacks}: When a summon gains this ability, choose an energy status. When you summon this creature, you may attach a rider to it’s next attack that inflicts the chosen elemental status.
\end{itemize}

\subsection*{Fourth \Facet}
\textbf{Evolution}: Each summon on your list gains a tactical ability or two features.

\subsection*{Fifth \Facet\perk}
\textbf{Enhanced Body}: Each summon on your list gains one of the following abilities:
\begin{itemize}
\thing \textbf{Elemental Body}: When a summon gains this ability, choose an elemental status. The first time that this summon is damaged by an enemy adjacent to it, that enemy gains the chosen elemental status.
\thing \textbf{Retributive}: When this summon falls below 0 HP, each enemy within 2 hexes of it become [Prone].
\thing \textbf{Fear Aura}: Whenever an enemy would move into a hex adjacent to this summon, that enemy must spend an additional movement point. This is a [Fear; Slow] effect.
\thing \textbf{Eldritch Body}: The first time each encounter that an enemy has line of sight to a summon with this ability, that enemy becomes [Dizzy 2] for one round.
\thing \textbf{Invisible}: This summon is [Invisible 3]. This effect does not end if the summoned creature attacks, as long as no targets of that attack are outside radius 3 of the summoned creature.
\thing \textbf{Glorious Form}: When you summon this creature, you may have each hex within 6 hexes of it become lit. This effect ends when the summoned creature dies or is dismissed. Whenever an ally within this area would be subject to a [Restorable; lesser] effect, if that effect is also a [Fear] effect, there is a 25\% chance that that effect is negated.\end{itemize}

\subsection*{Sixth \Facet}
\textbf{Evolution}: Each summon on your list gains a tactical ability or two features.

\subsection*{Seventh \Facet\perk}
\textbf{Master Summoner}: Once per encounter, you may summon a new creature that is not on your list. \emph{(Tip: Write up the summon while it is not your turn)}

\end{multicols}

%Minor
\section{Two Weapon Fighting}
\textbf{Minor Offense}

Fighting with two weapons has a long tradition. Assassins striking with a pair of daggers, pirates raiding their foes with a cutlass and flintlock, Cowboys with a six shooter in each hand, even battlemages who would swing a sword while raining fire on their foes all found value in a second weapon. 

\subsection*{Adaption}
This is a very general concept which can be used in many ways. Besides the obvious characters with 2 weapons, you can use it for someone who attacks with their offhand while doing something else with their main attention, such as casting spells, healing, summoning, etc. It can also be used with an unarmed character to represent them attacking extra fast, or an old 1-2. 
\begin{multicols*}{2}
\subsection*{First Facet\perk}
\textbf{Dual-Wield}: \emph{You have decided the best use of your other hand is to use a weapon.} You may wield 2 weapons at the same time. You may choose either weapon to make your major action with for the round, and all abilities from this track will use the other weapon, referred to as your offhand weapon.  Whenever you draw or switch weapons, you may draw or switch your offhand weapon as well.

\textbf{Offhand Blow} [W]:  You may attack an opponent at +1 to hit. This deals $3\times C\times Lvl$ damage on a hit.

\textbf{Offhand Strike}: You may use a surge with your offhand weapon. 

\subsection*{Second Facet}
\textbf{Combo Power}: You gain one of the following abilities each round, based on which types of weapon you are wielding. If you do not use your major action to attack with your weapon, you gain one of the other abilities based on your offhand weapon.
\begin{itemize}
\thing \textbf{Two Ranged Weapons} [W]: \emph{You can aim your weapons at two different targets effectively} As a minor action, when you attack with your major action, you may use it again; this cannot include targets that were targeted by the first action or other uses of this ability this round. You cannot use this with abilities that have the [S] tag. \textbf{Boost}: You may boost this major action.
\thing \textbf{Two Melee Weapons} [W]: \emph{You can create an opening for your attack.} As a minor action, your target takes a penalty against your major action, and [Vulnerability] to your attacks equal to your $C \times lvl$.
\thing \textbf{One Ranged and One Melee Weapon}: \emph{You can use your melee weapon to defend yourself while firing a ranged weapon.} You do not provoke reactions while firing a ranged weapon out of melee. If you hit with a melee attack, you can move the opponent 3 hexes away from you.
\thing \textbf{One Melee Weapon and a Non-Weapon Major} [W]: \emph{Your sword acts as a defense to ward off attackers while you work your magic} You do not provoke reactions from foes you target with your offhand weapon by using reckless abilities. Additionally, as a minor action you may strike everyone in melee range for $2\times C\times Lvl$ damage on a hit.
\thing \textbf{One Ranged Weapon and a Non-Weapon Major}: \emph{You force your opponents to dodge out of the way of your attacks} If you miss with your offhand attack, you may move its target 1 hex in any direction
\end{itemize}

\subsection*{Third Facet\perk}
\textbf{Dual Strike}: As a minor action, you may strike one target with your main weapon and another target with your offhand weapon, each at a +1 to hit, dealing $2\times C\times Lvl$ damage.

\textbf{Matched Pairs} : For each type of weapon you carry, you may carry an additional weapon of that type without counting against your max number of weapons.

\textbf{Doubled Reflexes}: You may take 1 Reaction with each weapon, if that weapon is capable of making reactions. You may still only take 1 reaction against any given action.

\textbf{Extreme Nimbleness}: If you wield 2 nimble weapons, you may avoid a reaction from moving even if you started in an enemy's melee range, as long as you were wielding both weapons at the start of your turn.

\subsection*{Fourth Facet}
\textbf{Deftness}: You gain one of the following abilities based on your offhand weapon:
\begin{itemize}
\thing \textbf{Melee}: Your offhand blow may also be a push, grapple, trip, or disarm attempt in addition to its damage.
\thing \textbf{Ranged}: The target of your off-hand blow is [Flat-footed]
\end{itemize}

\subsection*{Fifth Facet\perk}
\textbf{Rend}: \emph{The combined fury of your weapons leaves your opponents bleeding} Your offhand blow inflicts [Bleeding $C\times Lvl$] if it targets a creature targeted by your major action.

\textbf{Specialized Maneuvers}: Once per encounter, you may use one of the following abilities depending on your offhand weapon
\begin{itemize}
\thing \textbf{Corkscrew Dash} [W]: As a minor action while using a melee offhand weapon, you may dash 6 hexes, and strike every opponent within melee range of any hex you passed through. This deals $2\times C\times Lvl$ damage on a hit
\thing \textbf{Spray} [W]: As a minor action while wielding a ranged offhand weapon, you may strike 3 opponents. The same opponent may be targeted multiple times This deals $0.5 \times C \times Lvl$ damage per hit, and each target hit gains [Ongoing $C\times Lvl$]
\end{itemize}

\subsection*{Sixth Facet}
\textbf{X Defense}: \emph{You brace your weapon against your offhand weapon to steel yourself from oncoming attacks} As an immediate minor action, you may gain $2 \times C\times Lvl$ barrier hp. 

\subsection*{Seventh Facet\perk}
\textbf{Twinned Attack}: You focus both your weapons and attention to the task at hand
As a minor action, you may draw a offhand weapon that matches your main weapon without provoking, and gain 2 boosts to your major action and gain a rider that inflicts [Bleeding $C\times Lvl$] and [Slow 2] on a hit.
\end{multicols*}
\section{Acrobat}
\textbf{Minor Mixed}

While others may prefer to use armour or magic to protect them from weapons, you prefer not to be where the enemy's weapon is.

\subsection*{Adaption}
This track can be used to represent anything that is highly nimble and capable of dodging attacks.

\begin{multicols*}{2}
\subsection*{First \Facet\perk}
\textbf{Get Outta Dodge}: As an immediate minor action when an enemy goes to attack you, you can dash up to 2 hexes +1 hex per facet and become [Guarded]. Your base dodge chance is 10. Roll a d20; if you roll your dodge chance or lower, you move before the ability hits, the ability misses, and the opponent cannot retarget it. Otherwise, you are too slow and move after the ability hits you. If the attack is [Overwhelming], you don’t become [Guarded].

\textbf{Distraction}: As a minor action, you can distract 2 enemies within 12 hexes. They are [Off balance] and gain a penalty to their next major action. If the next major action they take is against you, you gain a +3 to your dodge chance against that action.

\subsection*{Second \Facet\perk}
\textbf{Instinctive Dodge}: You may use Get Outta Dodge while [Off Balance].
\textbf{Evasion}: If an effect which does partial damage on a miss misses you, or you succeed a defense roll for half damage, you may avoid the damage as an immediate minor. If you do so, you gain [Vulnerability $\frac{1}{2} \times C \times Lvl$] . If the effect targets an area, you instead gain  $\frac{1}{2} \times C \times Lvl$ temporary hp. 

\subsection*{Third \Facet\perk}
\textbf{Backflip}: If you have not been targeted by an enemy since the end of your last turn, you may dash 3 hexes as the start of your turn

\textbf{On Their Toes}: If you dodge an attack with Get Outta Dodge, the attackers is [off balance]

\subsection*{Fourth \Facet\perk}
\textbf{Vertical Dodge}: You may jump 2 when using Get Outta Dodge

\subsection*{Fifth \Facet\perk}
\textbf{Sidestep}: If an opponent would make a reaction against you, you may spend 1 mp to dash 1 hex before they can target you.

\subsection*{Sixth \Facet}
\textbf{Slip Through the Shadows}: Once per encounter, Get Outta Dodge may be a [Teleport].

\textbf{Counter}: Once per encounter, when you using Get Outta Dodge ability, if the attacker is adjacent to you and the dodge is successful, you may forgo the movement to have the attack simply miss and knock the attacker back one hex and disarm them.

\subsection*{Seventh \Facet\perk}
\textbf{Cheaters Dodge}: Once per [Encounter], as a minor action, you can roll a d10 instead of a d20 next time you use Get outta Dodge, and become [Guarded 2] instead of 1 unless it's against an [Overwhelming] ability.
\end{multicols*}

\section{Lawbearer}
\textbf{Minor Mixed}

You set the law and you enforce it.

\subsection*{Adaption}
This track can be used to represent \emph{??}.

\begin{multicols*}{2}
\subsection*{First \Facet\perk}
\textbf{Set Law}: As a minor action, choose an enemy within 12 hexes of you and a law. During their next turn, that enemy may choose to follow the limitations of the chosen law. If they do not, they become a [Lawbreaker]. The following is a list of available laws. You may treat [Lawbreakers] as having a target priority one stage higher.
\begin{itemize}
\thing \textbf{Minor Denial}: The chosen enemy cannot take minor actions.
\thing \textbf{Off Limits}: Choose 3 hexes, plus one additional hex for every level you possess. The chosen hexes cannot be adjacent to the adjacent creature. The chosen creature cannot enter the chosen hexes, and cannot draw line of sight through those hexes.
\thing \textbf{Mirror}: Choose a [Restorable; Greater] and a [Restorable; Lesser] effect currently you. The chosen creature gains those effects.
\end{itemize}

\textbf{Punish Lawbreakers}: As a minor action, you can use two surges against a [Lawbreaker]. That creature is no longer a [Lawbreaker].

\textbf{Shackle Lawbreaker}: As a minor action, choose a [Lawbreaker] within 12 hexes of you. That creature stops being a [Lawbreaker] and gains shackles. The shackles are an [Equipped] effect, and have $2\times C\times Lvl$ HP, but are not [Bonded]. The shackles disappear after two turns. At the end of each of your turns, while the shackles last, the shackled creature becomes [Weakened $2 \times C\times Lvl$].

\textbf{Debt Repaid}: When a [Lawbreaker] becomes [Wounded], you gain a refresh.

\subsection*{Second \Facet\perk}
\textbf{The Watchful Eye of the Law}: You gain [Scanner 12], however, all creatures gain immunity to this. Whenever a creature becomes a [Lawbreaker], that creature loses this immunity for the rest of the encounter.


\subsection*{Third \Facet\perk}
\textbf{Escape Fee}: For each enemy, once per encounter, when that enemy would stop being a [Lawbreaker], you may choose a [Restorable; Lesser] effect currently affecting you. You lose that condition, and that enemy gains that condition with the remaining duration.
\textbf{Strict Guidelines}: Once per encounter, as a minor action, choose an enemy. That enemy may immediately make a triggered strike or take a minor action with all decisions made by you. If they do not, they become a [Lawbreaker].

\subsection*{Fourth \Facet\perk}
\textbf{Criminal Record}: Whenever a creature becomes a [Lawbreaker], if it is not the first time in the encounter that it has become a [Lawbreaker], it becomes [Stopped] for one round.

\subsection*{Fifth \Facet\perk}
\textbf{Police Brutality}: Once per encounter, choose an enemy within 12 hexes. You may spend a minor action to have that enemy become [Unguarded 2] and their target priority increases by two stages. That enemy may choose to negate this by becoming a [Lawbreaker].

\subsection*{Sixth \Facet\perk}
\textbf{Leniency}: As an immediate minor action, when an enemy would become a [Lawbreaker], you may choose for that enemy to not become a [Lawbreaker]. If you do, you may take an additional major action during your next turn.

\subsection*{Seventh \Facet\perk}
\textbf{Profiling}: Once per encounter, as a minor action, choose an enemy within 12 hexes. That enemy becomes a [Lawbreaker].

\end{multicols*}
%Passive
\section{Hunter’s Marks}
\textbf{Passive Offense}

You combine reactants which, individually, may be weak, but together, combine for a flexible array of powerful effects.

\subsection*{Adaption}
This track is, by default, flavoured as mixing ingredients, however, it can instead be fluffed as making your own spells.

\begin{multicols*}{2}
\subsection*{First \Facet\perk}
\textbf{Magic Markings}: At the beginning of each encounter, you gain four hunter's marks.  At the beginning of each round, you may choose a creature within radius 24. That creature becomes [Marked] for one round, and becomes [Vulnerable $2\times C\times lvl$] to the next damage they receive from a creature with the [Hunter] tag. Additionally, creatures with the [Hunter] tag ignore target priority when attacking creatures with the [Marked] tag. This ability ignores target priority. Refresh: you gain a hunter’s mark and may apply an extra one on your turn.

\textbf{Magic Hunter}: At the beginning of your turn, if none of your allies have the [Hunter] tag, you may choose an ally within radius 24. That ally receives the [Hunter] tag until they become [Wounded] or until they choose to end the status.

\textbf{Hunter’s Traps}: When a creature with the [Hunter] tag leaves a hex, they may spend an additional movement point. If they do, that hex becomes [Afflicted]. When an enemy enters a hex [Afflicted] this way, that enemy becomes [Stopped] until the end of their turn. 

\textbf{Dying Shot}: When an ally with the [Hunter] tag would become [Wounded], you may spend any number of hunter's marks. For each hunter's mark spent this way, the enemy that last damaged the ally with the [Hunter] tag becomes [Vulnerable $2\times C\times lvl$] to the next source of damage dealt by an ally with the [Hunter] tag. Additionally, you may spend a single hunter's mark after doing this. If you do, the ally with the [Hunter] tag may a surge on the creature who last damaged them.

\subsection*{Second \Facet\perk}
\textbf{Prowling Hunter}: Allies with the [Hunter] tag gain an additional two movement points each turn. 

\textbf{Hunter’s Resistance}: Allies with the [Hunter] tag ignore [Difficult Terrain].

\subsection*{Third \Facet\perk}
\textbf{Hunter’s Strength}: Allies with the [Hunter] tag are immune to [Slow].

\textbf{Hunter’s Presence}: Enemies with the [Marked] tag are inflicted with [Fear; Slowed 1] for as long as they are [Marked].

\subsection*{Fourth \Facet\perk}
\textbf{Hunter’s Endurance}: Allies with the [Hunter] tag are immune to [Bleeding].

\textbf{Hunter’s Knowledge}: Allies with the [Hunter] tag can spend an extra movement point before they enter a hex that is [Afflicted] and remove [Afflicted] from it.

\subsection*{Fifth \Facet\perk}
\textbf{Hunter’s Gaze}: Once per encounter, you may choose a creature with the [Marked] status. That creature becomes [Fear; Stopped] for one round.

\textbf{Hunter’s Concentration}: Allies with the [Hunter] tag are immune to [Dizzy].

\subsection*{Sixth \Facet}
\textbf{Hunter’s Stealth}: At the end of their turn, all allies with the [Hunter] tag are treated as [Stealth 2] to creatures that are [Marked]. 

\subsection*{Seventh \Facet\perk}
\textbf{Ultimate Mark}: Once per encounter, choose a creature. You may ignore cover and concealment when using Magic Markings on that creature for one round. Allies with the [Hunter] tag may ignore cover and concealment when attacking that creature for one round.

\textbf{Hunter’s Poison}: When you use Hunter’s Trap, enemies who enter the [Afflicted] hex receive [Ongoing $C\times lvl$].

\end{multicols*}

\section{Rage}
\textbf{Passive Offense}

Characters with the rage aspect have learnt to master their rage and use it to annihilate their enemies.

\subsection*{Adaption}
Oddly enough, in some ways, this track rewards a sense of ``honour''. You attack people who attacked you first. As a result, this can be used to represent an honourable fighter.



\begin{multicols*}{2}
\subsection*{First \Facet\perk}
\textbf{Don't Tick Me Off}: \emph{You get angry at the slightest grievance.} Whenever an enemy tries to target you with an offensive ability or you make a reactive strike against an enemy, you may treat that enemy as a [Victim] until the end of your next turn. If, at the beginning of your turn, there are no [Victims] within 12 hexes, the nearest enemy to you becomes a [Victim] until the end of your turn. You may treat [Victims] as though their target priority was one stage higher.

\textbf{Muscle Drain}: You begin each encounter with no muscle drain. Once per round, if you have less than four muscle drain, you can use one of the following abilities. When you do, you gain a point of muscle drain after using the ability.
\begin{itemize}
\thing \textbf{Angry Attack}: When you damage a [Victim], you may increase the damage by $2 \times C \times lvl$ and make a rider attack. If the rider attack hits, the [Victim] is moved two hexes away from you. This distance is reduced by one for every two points of muscle drain you have. If you damaged the enemy with a melee weapon, that [Victim] is instead moved four hexes away from you. This distance is instead reduced by one for every point of muscle drain you have. 
\thing \textbf{Friendly Fury}: When you use an action that affects one or more allies, choose one of the affected allies, and an enemy who is currently a [Victim]. The next time that the chosen ally deals damage to that enemy, they may increase the damage by $2 \times C \times Lvl$ and make a rider attack. If the rider attack hits, the [Victim] is moved two hexes away from the ally. This distance is reduced by one for every two points of muscle drain you possess. If this ability was triggered by damage with a melee weapon, the enemy is instead moved four hexes away from the ally. This distance is reduced by one hex for every point of muscle drain you have.
\end{itemize}

\textbf{Refresh}: As a refresh, you may lose one point of muscle drain. If you have no muscle drain, prevent the next point of muscle drain you would gain. Additionally, you may use Angry or Friendly Fury an additional time during either your current or next turn.

\textbf{Final Ferocity}: When you would become [Wounded], you may use four surges against the last enemy to deal you damage. For each point of muscle drain you have, you use one less surge with this ability. You may switch your weapon before each surge. When you do this, you gain one point of muscle drain for each surge you used.

\subsection*{Second \Facet\perk}
\textbf{No Escape}: \emph{When you're ticked off, there is no escape from your wrath.} For each [Victim], the first time each round that that [Victim] would move away, you may dash four hexes and use a surge against them. The distance you can dash is reduced by one hex for each point of muscle drain you have. If a creature moves away from you, becomes a [Victim], and then moves away from you again, that still triggers this ability. Enemies are aware of this effect, even before any enemy becomes a [Victim].

\subsection*{Third \Facet\perk}
\textbf{Fear the Beast}: \emph{If someone hasn't drawn your ire, it is in their best interest to keep it that way.} Whenever an enemy who is not a [Victim] moves into or our of a hex adjacent to you, that enemy may become [Slow X], where X is 4 minus the amount of muscle drain you possess. If they do not, that enemy becomes a [Victim] and becomes [Vulnerable $C\times lvl$] to the next attack made against them by you.

\subsection*{Fourth \Facet\perk}
\textbf{Moment to Rest}: When you would use a surge from an ability outside of this track, you may instead lose one point of muscle drain.

\subsection*{Fifth \Facet\perk}
\textbf{Beg for my Freaking Forgiveness, Scum}: \emph{Enemies who have slighted you should be down on their knees, begging for your mercy.} When you use Angry Attack, you may also make a rider attack against that [Victim]. When you use Friendly Fury, the affected ally may make a rider attack against the chosen enemy. Either way, if the rider attack hits, that [Victim] becomes [Prone].

\subsection*{Sixth \Facet\perk}
\textbf{Reckless Abandon}: Once per round, when you take a major action that targets a [Victim], you may choose to gain a boost. If you do, you become [Vulnerable $C \times level$]. Once per round, when an ally who you have affected with Friendly Fury during your last turn takes a major action that targets a [Victim], that ally may choose to gain a boost. If they do, they become [Vulnerable $C \times level$].

\subsection*{Seventh \Facet\perk}
\textbf{You What, Mate?}: Whenever an enemy takes an action within 12 hexes of you, you may have that enemy become a [Victim] until the end of your next turn.
\end{multicols*}
\section{Battle Bonds}
\textbf{Passive Varied}

You are your partner can face any foe, as long as you’re together.

\subsection*{Adaption}
This track can be used to represent the power of true love, an oath to protect your partner, or linked twins. 

\emph{Sidebar: Some of the abilities are shared by both, such as Share HP and it’s effects, and others are usable only by the character with this track, not the partner - the one with this track has a higher investment in the partnership and has more abilities relating to that.}

\begin{multicols*}{2}
\subsection*{First \Facet\perk}
\textbf{Thick and Thin}: When you gain this \facet{}, choose an ally. That ally becomes your partner. You are not their partner unless they also possess this \facet and designate you as their partner.

\textbf{Right Behind You!}: Twice per encounter, you can raise or lower the target priority of your partner by 1. If an enemy violates this target priority, they gain [Vulnerable $C\times lvl$] to the next attack made by you or your partner.

\textbf{Partnered Benefits}: Choose one of the following. You gain that ability. This choice is permanent.
\begin{itemize}
\thing \textbf{Shared Skills}: You begin each encounter with four [Partner] tokens. Once per round, you may spend a [Partner] token to use a surge, or to allow your partner to use a surge. When using this ability, you may use your ally's surge ability, and they may use yours. This can be done during either your turn or your partner's turn. Additionally, you may choose a [Vengeance] ability. \textbf{Refresh}: As a refresh, gain an additional [Partner] token. You may use this ability an additional time during your next turn.
\thing \textbf{Duo}: You begin each encounter with four [Partner] tokens. Once per round, you may spend a [Partner] token to give you and your ally a boost.  Additionally, you may choose a [Vengeance] ability. \textbf{Refresh}: As a refresh, gain an additional [Partner] token. You may use this ability an additional time during your next turn.
\thing \textbf{Warmth}: Twice per encounter, you may heal your partner by $4\times C\times Lvl$, or your ally may heal you by $4\times C\times Lvl$. Excess healing becomes Temporary Hit Points. You do not gain a [Vengeance] ability. \textbf{Refresh}: As a refresh, you may heal yourself or your partner by $2\times C\times Lvl$. Excess healing becomes Temporary Hit Points.
\thing \textbf{Shared Stoneskin}: At the beginning of each encounter, you and your partner each gain $4\times C\times Lvl$ resistance. \textbf{Refresh}: As a refresh, you may gain $2\times C\times Lvl$ resistance, or you may have your ally gain $2\times C\times Lvl$ resistance.
\end{itemize}

\textbf{Vengeance}: If you have the Shared Skills ability or the Duo ability, choose one of the following. You gain that ability. This choice is permanent.
\begin{itemize}
\thing \textbf{Retribution}: While your partner is [Wounded], you may spend a [Partner] token to use a surge. You may use your partner's surge ability instead of your own. While you are [Wounded], your partner may spend one of your [Partner] tokens to use a surge. They may use one of your surge abilities instead of their own.
\thing \textbf{Fury}: While your partner is [Wounded], you may spend a [Partner] token to become [Energized $2\times C\times Lvl$]. While you are [Wounded], your partner may spend a [Partner] token to become [Energized $2\times C\times Lvl$].
\end{itemize} 
\subsection*{Second \Facet\perk}
\textbf{Share HP}: Once per turn, you may take an amount of damage not exceeding a quarter of your current HP. Your partner heals an amount equal to the damage you took this way. Once per turn, your partner may take an amount of damage not exceeding a quarter of their current HP. You heal an amount equal to the damage they took this way.

\textbf{Lockstep}: You and your partner may spend double the movement points when entering a hex. If either of you do this, the other may move one hex in the same direction. Additionally, choose either Dizzy, Stop, Slow or Prone. This choice is permanent While you are within radius 3 of your partner, you and your partner are both immune to the chosen condition, as well as forced movement.

\subsection*{Third \Facet\perk}
\textbf{Furious Vengeance}: You only gain this ability if you possess either Retribution or Fury. Once per round, while your partner is [Wounded], when you would make an attack, that attack gains a rider. Once per round, while you are [Wounded], when your partner would make an attack, that attack gains a rider. In both circumstances, the rider automatically hits and either inflicts [Prone] or pushes the enemy one hex.

Choose one of the following:
\begin{itemize}
\thing \textbf{Pull It Together}: When you or our partner uses Share HP, choose a [Restorable; Lesser] condition affecting the creature healed through Share HP. End that condition.
\thing \textbf{Scissor}: When you deal damage to an enemy, if your partner has also dealt damage to that enemy since the beginning of your last turn, that enemy begins [Bleeding $C\times Lvl$]. When your partner deals damage to an enemy, if you also dealt damage to that enemy since the beginning of your partner's last turn, that enemy begins [Bleeding].
\thing \textbf{Take My Hand}: If you have a movement type (swim, fly, burrow, jump, teleport) you can allow your partner to also have it while adjacent to them.
\end{itemize}

\subsection*{Fourth \Facet}
\textbf{Anything for you}: Once per round, when your partner would take damage, you may choose to take that damage instead. This counts as a use of Share HP.

\textbf{Perfect Communication}: You and your partner can always communicate perfectly and, if willed, without being overheard, as long as you have line of sight to each other. This can be telepathy, hand signs, code phrases, or other.

\textbf{Shared Senses}: You and your partner may draw Line of Sight as though you were in either your hex or your allies hex.

\subsection*{Fifth \Facet}
Choose one:
\begin{itemize}
\thing \textbf{Shooting Star} Once per encounter, you may teleport to any hex within range 3 of your partner, or your partner may teleport to a hex within range 3 of you. Teleportation made as part of this ability has the [Warp] descriptor.
\thing \textbf{Fearless}: While you and your partner are within radius 6 of each other, you and your partner are both immune to fear.
\end{itemize}

\subsection*{Sixth \Facet}
\textbf{Riotous Vengeance}: If you or your partner is wounded and you possess a Vengeance ability, attacks made by you or your partner gain one of the following riders.
\begin{itemize}
\thing \textbf{Push}: This rider automatically hits. A creature hit by this rider is moved two hexes.
\thing \textbf{Trip}: This rider automatically hits. A creature hit by this rider becomes [Prone]. At the beginning of their next turn, they become Off Balance.
\end{itemize}

\textbf{Not Alone}: Once per encounter, when you are adjacent to your partner, you may use Share HP on them while they are [Wounded]. When used this way, if your partner's HP increases to above 0, they stop being [Wounded].

\subsection*{Seventh \Facet\perk}
\textbf{Together Forever}: You may spend a movement point to give your partner a movement point. Your partner may spend a movement point to give you a movement point. You and your partner may both move during both your turn and your partner's turn.

Choose one:
\begin{itemize}
\thing \textbf{Tag Team} Once per round, during your turn, you may teleport to your partner's hex. When you do, your partner teleports to your hex without provoking reactions. Once per round, during your partner's turn, your partner may teleport to your hex. When your partner does this, you teleport to your partner's hex. All movement made with this ability has the [Warp] descriptor.
\thing \textbf{Harmonic Movement}: When you or your ally would take their turn, you may both take your turn simultaneously.
\end{itemize}
\end{multicols*}


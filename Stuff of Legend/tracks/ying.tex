\section{Ying}
\textbf{Major Defensive}

You channel both the rift of death and the rift of life. While one may seem to be dominant, both always work in harmony in your abilities.

\subsection*{Adaption}
This can represent any kind of healer, whether they are casting spells, imparting the favor of their gods, or shooting vials of adrenaline into people.

\begin{multicols*}{2}
\subsection*{First \Facet\perk}
\textbf{Cure Wounds} [RmS]: As a major action, you may heal an ally within twelve hexes for $2 \times C \times Lvl$. When you do so, you may add one bonus to them. \textbf{Boost}: Heal an additional $C \times Lvl$.

\textbf{Energiser}: You start with the energiser bonus. When you use this bonus, the target of your cure wounds is [Energized $2 \times C \times Lvl$].

\textbf{Inflict Wounds} [RmS]: As a major action, choose an enemy within 12 hexes. That enemy makes a defence roll. If they fail the defence roll, deal 1d4 damage for every level you possess, multiplied by your caliber to that enemy. If they instead succeed, deal half that damage instead. Regardless of success or failure, add a hindrance to the targeted enemy. This ability has any tags that the hindrance has. \textbf{Boost}: After the effects of the hindrance, deal an additional $C \times Lvl$ damage. This damage is not halved on a successful defence roll.

\textbf{Weakening Influence} [O]: You start with the weakening influence hindrance. When you use this hindrance, the target of your inflict wounds is [Weakened X], where X is the amount of damage dealt by the inflict wounds ability.

\textbf{Surge} [Rm]: As a surge, you may use the inflict wounds ability. You may not add a hindrance when using it this way.


\subsection*{Second \Facet\perk}
\textbf{Relief}: When you use cure wounds on an ally, you may also choose a [Restorable; Lesser] condition affecting that ally. If you do, end that condition.

\textbf{Extended Protection}: If you heal a damaged ally to more than their maximum hit points, the excess becomes temporary HP.

\textbf{Flaming Sword}: You gain the flaming sword bonus. When you use this bonus, choose an [Energy Condition]. The target of your cure wounds may attach a rider to the first major action they take during their next three turns. This rider deals $C \times Lvl$ damage on a hit and the target receives the chosen [Energy Condition]. If the rider misses, the target of the attack still takes half damage.

\textbf{Vampiric} [O]: You gain the Vampiric Wounds hindrance. When you use this hindrance, you heal yourself an amount equal to the damage dealt by inflict wounds.

\subsection*{Third \Facet\perk}
\textbf{Protective Aura}: You gain the Protective Aura bonus. When you use this bonus, the next time that the affected creature would be damaged, the creature who damaged the affected creature takes damage equal to the amount healed by cure wounds.

\textbf{Doom}: You gain the Doom hindrance. When you use this hindrance, the target of your inflict wounds ability gains [Persistent $2\times C\times Lvl$] if they fail the defense roll.

\subsection*{Fourth \Facet\perk}
\textbf{Vengeance}: You gain the Vengeance bonus. When you use this bonus, the targeted ally may move up to three hexes and use a surge against an opponent that attacked them in the last round.

\textbf{Parasite}: You gain the Parasite hindrance. When you use this hindrance, the target of your inflict wounds ability gain an equipped effect with $2 \times C \times Lvl$ non-bonded HP.  While the equipped effect remains alive, the affected creature is [Slowed 3].

\subsection*{Fifth \Facet\perk}
\textbf{Desperate Cure}: You may use cure wounds as an immediate major action.

\textbf{Quick Death}: You may use inflict wounds as an immediate major action.

\textbf{Healing Surge}: You gain the Healing Surge bonus. When you use this bonus, the target of your cure wounds ability may take a minor action.

\textbf{Hammer}: You gain the Hammer hindrance. When you use this hindrance on an enemy, if that enemy is making an attack, make an attack with a -2 penalty to hit. If hit hits, the enemy is moved 1 hex per odd circle before their attack (potentially wasting their action). If the attack misses, the movement still occurs, but it occurs after their attack.

\subsection*{Sixth \Facet\perk}
\textbf{Glory of the Fallen}: \emph{Your healing is so powerful it can revive wounded allies.} You may use cure wounds on a [Wounded] creature. If this brings them above 0 hp, they are no longer wounded.

\textbf{Damage Aura}: You gain the Damage Aura bonus. When you use this bonus, the target of your cure wounds ability deals $\frac{1}{2} \times C \times lvl$ damage to each creature within their melee range at the beginning of each of their turns for the rest of the encounter.

\textbf{Weakening Burst}: You gain the Weakening Burst hindrance. When you use this hindrance, the target of your inflict wounds and each enemy adjacent to them is [Weakened $-|-|7|8 \times C \times Lvl$].

\subsection*{Seventh \Facet\perk}
\textbf{Lifelink}: You may use cure wounds on your allies even if you do not have line of effect to them. They must still be within range 12 of you.

\textbf{Etheral}: You gain the Ethereal bonus. When you use this bonus, the target of your cure wounds has [Phasing] and a lowered target priority for one round, and they gain two boosts. Attacks against this target for one round that violate target priority receive a penalty.

\textbf{Binding}: You gain the Binding hindrance. When you use this hindrance, the target of your inflict wounds is [Slowed 3] for one round and receive two penalties.
\end{multicols*}

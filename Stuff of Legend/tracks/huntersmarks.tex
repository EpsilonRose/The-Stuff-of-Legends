\section{Hunter’s Marks}
\textbf{Passive Offense}

You combine reactants which, individually, may be weak, but together, combine for a flexible array of powerful effects.

\subsection*{Adaption}
This track is, by default, flavoured as mixing ingredients, however, it can instead be fluffed as making your own spells.

\begin{multicols*}{2}
\subsection*{First \Facet\perk}
\textbf{Magic Markings}: At the beginning of each encounter, you gain four hunter's marks.  At the beginning of each round, you may choose a creature within radius 24. That creature becomes [Marked] for one round, and becomes [Vulnerable $2\times C\times lvl$] to the next damage they receive from a creature with the [Hunter] tag. Additionally, creatures with the [Hunter] tag ignore target priority when attacking creatures with the [Marked] tag. This ability ignores target priority. Refresh: you gain a hunter’s mark and may apply an extra one on your turn.

\textbf{Magic Hunter}: At the beginning of your turn, if none of your allies have the [Hunter] tag, you may choose an ally within radius 24. That ally receives the [Hunter] tag until they become [Wounded] or until they choose to end the status.

\textbf{Hunter’s Traps}: When a creature with the [Hunter] tag leaves a hex, they may spend an additional movement point. If they do, that hex becomes [Afflicted]. When an enemy enters a hex [Afflicted] this way, that enemy becomes [Stopped] until the end of their turn. 

\textbf{Dying Shot}: When an ally with the [Hunter] tag would become [Wounded], you may spend any number of hunter's marks. For each hunter's mark spent this way, the enemy that last damaged the ally with the [Hunter] tag becomes [Vulnerable $2\times C\times lvl$] to the next source of damage dealt by an ally with the [Hunter] tag. Additionally, you may spend a single hunter's mark after doing this. If you do, the ally with the [Hunter] tag may a surge on the creature who last damaged them.

\subsection*{Second \Facet\perk}
\textbf{Prowling Hunter}: Allies with the [Hunter] tag gain an additional two movement points each turn. 

\textbf{Hunter’s Resistance}: Allies with the [Hunter] tag ignore [Difficult Terrain].

\subsection*{Third \Facet\perk}
\textbf{Hunter’s Strength}: Allies with the [Hunter] tag are immune to [Slow].

\textbf{Hunter’s Presence}: Enemies with the [Marked] tag are inflicted with [Fear; Slowed 1] for as long as they are [Marked].

\subsection*{Fourth \Facet\perk}
\textbf{Hunter’s Endurance}: Allies with the [Hunter] tag are immune to [Bleeding].

\textbf{Hunter’s Knowledge}: Allies with the [Hunter] tag can spend an extra movement point before they enter a hex that is [Afflicted] and remove [Afflicted] from it.

\subsection*{Fifth \Facet\perk}
\textbf{Hunter’s Gaze}: Once per encounter, you may choose a creature with the [Marked] status. That creature becomes [Fear; Stopped] for one round.

\textbf{Hunter’s Concentration}: Allies with the [Hunter] tag are immune to [Dizzy].

\subsection*{Sixth \Facet}
\textbf{Hunter’s Stealth}: At the end of their turn, all allies with the [Hunter] tag are treated as [Stealth 2] to creatures that are [Marked]. 

\subsection*{Seventh \Facet\perk}
\textbf{Ultimate Mark}: Once per encounter, choose a creature. You may ignore cover and concealment when using Magic Markings on that creature for one round. Allies with the [Hunter] tag may ignore cover and concealment when attacking that creature for one round.

\textbf{Hunter’s Poison}: When you use Hunter’s Trap, enemies who enter the [Afflicted] hex receive [Ongoing $C\times lvl$].

\end{multicols*}

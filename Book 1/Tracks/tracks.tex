\chapter{Tracks}
\begin{multicols*}{2}
Tracks come in a variety of different forms, however, a lot of common features can be noted. All tracks are either Major, Minor or Passive, and either Offensive, Defensive, Mixed or Varies. In this section, we give an overview of each type of track.

Major tracks determine how you use your major action. These tracks offer powerful options to use in place of basic attacks. Each character requires one of these. Examples include Compensator, which allows you to wield powerful weapons and Human Shield, which allows you to take powerful defensive stances to protect your allies and heal them. These tracks also give you your triggered strikes and may provide boosts to your charge attacks. As your major action is worth more than your other actions, these tracks often are about twice as powerful as other tracks, in exchange for requiring the use of your major action.

Minor tracks determine how you use your minor action. These tracks offer options to use in place of default minor actions. Each character requires one of these. Examples include Breath Weapon, allowing you to breath fire or other elements, and Shield, which allows the player to block incoming damage.

Passive tracks generally do not use actions, but may influence your movement. Once again, each character requires one of these. Examples include Death Armor, which allows you to protect yourself with the souls of your fallen enemies, and Ninja, which allows you to hide from your enemies.

Offensive tracks are generally focused on damaging your enemies, however, they can also amplify the damage-dealing abilities of your allies. Each party should have several players with at least one offensive track, however, a character without any offensive tracks is viable as a support character. Examples of offensive tracks include Elementalist, which allows you to throw powerful magic at your enemies, Two Weapon Fighting, which allows you to make off-hand attacks, and Fire Elemental, which burns creatures around you.

Defensive tracks are generally focused on keeping you and your allies alive. This can be done by giving you more HP or recovering HP you lose, or by reducing the ability of enemies to damage you. In some occasions, it can negate an enemy attack entirely. It is highly recommended that a player has at least one defensive track. A character with 3 defensive tracks is viable, however, they will exist mostly as support. Major defensive tracks will often also deal damage as well as provide defenses, however, they should not be the only form of damage that the party can deal.

Mixed tracks are tracks which contain some combination of offense and defense. While other tracks provide pure offense or pure defense, mixed tracks allow the player to adjust to the specific situation, generally at the cost of the more tactical abilities. Examples include the polymorph track, which allows the player to take on forms with potent offensive or defensive abilities as the situation requires, the potions track, which gives access to a wide variety of potions such as potions of might, potions of healing and potions of haste. A mixed track is not required. More experienced players can combine multiple mixed tracks to form a valid combination of offense and defensive, however, this is a difficult character to play well, and may often find himself dealing insufficient damage, or having insufficient ways to protect himself.

Varied tracks are tracks that have permanent choices that change the role or function of a track - unlike Mixed, they can’t adapt to the situation, but typically can cover their chosen role at full potency. A character can use a varied track as an offensive or defensive track by taking abilities that are mostly offensive or mostly defensive.

\end{multicols*}
